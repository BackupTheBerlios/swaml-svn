
\chapter{Metodolog�a}

Las circunstacias profesionales del alumno en el periodo que se realiz� el
proyecto hicieron que se tomaran de forma natural muchos de los m�todos de
las metodolog�as �giles, de manera que permitiera un �ptimo y flexible reparto 
de tiempos entre todas las tareas a desarrollar a lo largo del proyecto.

As� se termin� utilizando de manera natural muchas de las normas de este tipo
de m�todos:

\begin{itemize}
  \item Simplicidad: desde el primer momentos se busc� hacer el proyecto de la 
	manera m�s sencilla posible.
  \item Desarrollo iterativo e incremental: una vez se fue teniendo las 
	funcionalidades principales, construyendo alrededor de los casos de uso
	principales peque�as utilidades para enriquecer la aplicaci�n.
  \item Propiedad del c�digo compartida: desde las primeras lineas, todo el 
	c�digo ha estado publicado en el 
	Subversion\footnote{\url{http://subversion.tigris.org/}}  del proyecto, 
	al alcance en cualquier momento de las tres personas involucradas en el 
	proyecto.
  \item Refactorizaci�n del c�digo: de manera frecuente partes del c�digo del 
	proyecto han sido sometidas a t�cnicas de refactorizaci�n para mejorar 
	partes que no hab�a sido codificadas de la mejor de las maneras.
  \item Pruebas unitarias (FIXME: mentira, pero es una tarea que tengo que abordar 
	con PyUnit)
  \item Frecuente interacci�n del equipo de programaci�n con el cliente o usuario:
	en este caso la figura del cliente ha tenido la forma del co-director del
	proyecto (Diego Berrueta), con el que se han mantenido frecuentes encuentros
	para tratar aspectos concretos del proyecto.
\end{itemize}


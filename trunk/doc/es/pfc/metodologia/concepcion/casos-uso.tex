
\subsection{Especificación de casos de uso}\label{sec:espec-casos-uso}

Este documento contiene la descripción de los casos de uso. El modelo 
de casos de uso es un modelo de las funciones que realiza el sistema y 
su entorno, y sirve de contrato entre el cliente y los desarrolladores. 
Se emplea como entrada para las actividades de análisis, diseño y test.

Este documento contiene además aquellos requisitos que no pueden ser 
obtenidos tan solo con un análisis basado en casos de uso, como los 
requisitos de rendimiento o fiabilidad.

\subsubsection{Actores}

Un actor define un conjunto coherente de roles que los usuarios del 
sistema interpretan cuando interactúan con el mismo. Puede ser un 
individuo o un sistema externo.

Se procederá a enumerar los actores que participan en el sistema, 
así como una breve descripción de cada uno de ellos.

\begin{itemize}
  \item \textbf{Usuario:} Representa a la persona que interacturá con 
	la base del conocimiento generada. No manipulará los datos, 
	sino que los usará para hacer consultas y/o busquedas.
  \item \textbf{Administrador:} Representa al usuario que se encargue 
	de administrar el servicio y la máquina que lo aloje. Su labores 
	principales serán la de programar actualizaciones y verificar 
	que hayan resultado satisfactoriamente.
\end{itemize}

\subsubsection{Límites del sistema}

En la figura siguiente se establecen los límites del sistema, así 
como los principales casos de uso que posteriormente serán refinados 
en siguientes etapas del análisis.

\begin{figure}[ht]
 	\centering
	\includegraphics[width=11cm]{images/uml/caso-uso-general.png}
	\caption{Casos generales de uso}
	\label{fig:uml:casos-uso}
\end{figure}

Como se ve en la figura~\ref{fig:uml:casos-uso} se puede distinguir claramente
tres grandes bloques de casos de uso:

\begin{itemize}
 \item Publicar los datos
 \item Enriquecerlos
 \item Consultarlos
\end{itemize}

\subsubsection{Casos de uso}\label{sec:casos-uso}

Refinando el diagrama anterior se identifican varios casos de uso que paso a
enumerar:

\begin{itemize}
 \item Parametrizar el sistema
 \item Publicar
 \item Enriquecer los datos
 \item Consular los archivos generados
 \item Consultar la información extra generada
\end{itemize}

Para pasar a describirlos más detalladamente:

\begin{itemize}

  \item \textbf{Parametrizar el sistema:}
 	\begin{itemize}
 	  \item \textbf{Descripción:} Este caso de uso representa la labor 
		que el usuario administrador debe realizar para configurar 
		correctamente el sistema.
 	  \item \textbf{Flujo de eventos:} El caso de uso comienza cuando el
		usuario administrador comienza a editar una configuración, 
		bien manualmente o mediante el asistente que acompaña al 
		software.
	  \item \textbf{Precondiciones:} Es necesario disponer de un mailbox
		a exportar.
	  \item \textbf{Postcondiciones:} Ninguna detectada.
	\end{itemize}

  \item \textbf{Publicar:}
 	\begin{itemize}
 	  \item \textbf{Descripción:} Representa la acción de publicación 
		propiamente dicha.
 	  \item \textbf{Flujo de eventos:} El proceso es un proceso por 
		lotes que a partir de una configuración genera una serie 
		de ficheros RDF. Internamente se divide en varios procesos:
		\begin{enumerate}
		 \item Iniciar publicación
		 \item Imprimir estadísticas
		 \item Terminar publicación
		\end{enumerate}
	  \item \textbf{Precondiciones:} Disponer de una configuración correcta.
	  \item \textbf{Postcondiciones:} El directorio destino de la exportación
		debe poder \emph{consumirse} mediante otro servicio, como un 
		servidor \texttt{HTTP} (Apache o similar).
	\end{itemize}

  \item \textbf{Enriquecer los datos:}
 	\begin{itemize}
 	  \item \textbf{Descripción:} Representa la interacción del 
		sistema con otras bases del conocimiento externas, 
		principalmente los FOAF de los suscriptores a la lista 
		de correo, para enriquecer la información en determinados 
		aspectos.
 	  \item \textbf{Flujo de eventos:} Se tratar de un proceso que se 
		repite iterativamente con cada uno de los suscriptores:
		\begin{enumerate}
		 \item Buscar su FOAF
		 \item Si lo tiene:
		 \begin{enumerate}
		  \item	Enlazar al suscriptor con su FOAF
		  \item Consultar sus coordenadas geográficas
		  \item Consultar su foto
		 \end{enumerate}
		 \item Si no lo tiene continuar con el siguiente suscriptor
		\end{enumerate}
	  \item \textbf{Precondiciones:} Disponer de la lista de suscriptores 
		cargada en memoria.
	  \item \textbf{Postcondiciones:} Ninguna detectada.
	\end{itemize}

  \item \textbf{Consular los archivos generados:}
 	\begin{itemize}
 	  \item \textbf{Descripción:} Representa la interacción del 
		usuario con los datos generados. Desde una simple consulta 
		manual a los ficheros RDF generados, hasta realizar consultas 
		de una forma más sofisticada.
 	  \item \textbf{Flujo de eventos:} Ninguno particular.
	  \item \textbf{Precondiciones:} Disponer de la lista exportada a RDF.
	  \item \textbf{Postcondiciones:} Ninguna concreta.
	\end{itemize}

  \item \textbf{Consultar la información extra generada:}
 	\begin{itemize}
 	  \item \textbf{Descripción:} Este caso de uso representa la consulta 
		por parte del usuario de la información extra generada, por el 
		ejemplo los suscriptores en formato KML.
 	  \item \textbf{Flujo de eventos:}
		\begin{enumerate}
		 \item Consultar
		 \item Explotar los datos
		\end{enumerate}
	  \item \textbf{Precondiciones:} Disponer de la información geográfica 
		de los suscriptores.
	  \item \textbf{Postcondiciones:} Explotación de estos datos, por ejemplo
		visualizandolos\footnote{\url{http://maps.google.es/maps?q=http://swaml.berlios.de/demo/subscribers.kml}}
		con Google Maps\footnote{\url{http://maps.google.es/}}.
	\end{itemize}

\end{itemize}





\chapter{Propuesta inicial\label{sec:propuesta}}

Diego Berrueta (Agosto 2005)

Las listas de correo son parte fundamental de la comunicación en Internet.
Existen listas de correo dedicadas a cualquier tema de inter\'es imaginable.
Hoy en día, es común que las listas de correo publiquen sus archivos (los
mensajes antiguos) en forma de páginas web, lo que dispara su utilidad,
especialmente en combinación con los buscadores actuales. Gracias a esta 
publicación, es posible consultar los mensajes desde el navegador, sin 
necesidad de estar suscrito a las listas de correo, y también se puede 
localizar un mensaje usando Google u otro buscador.

Estos archivos contienen una formidable base de conocimiento, especialmente
en temas técnicos. Un uso muy común consiste en introducir un mensaje de error 
(de una aplicación) en Google\footnote{\url{http://www.google.es/}} y obtener 
como resultado un mensaje archivado que aborda el problema, probablemente 
porque alguien se ha encontrado previamente con el mismo error y ha efectuado 
la consulta en una lista de correo póblica. Con suerte, alguna de las respuestas 
al mensaje localizado contendrá la solución al problema, aportada por un experto 
suscrito a la lista de correo.

\section*{Problemas}

Por desgracia, consultar los archivos de una lista de correo en la web es
más incómodo que hacerlo mediante un cliente de correo electrónico.
Por poner sólo algunos ejemplos, el navegador no permite ejecutar ninguna
de estas acciones:

\begin{itemize}
 \item Mostrar el hilo de la conversación en forma de árbol.
 \item Imprimir el hilo completo.
 \item Mostrar una lista de los mensajes entre dos fechas arbitrarias.
 \item Ocultar los mensajes que no tienen respuestas.
 \item Mostrar sólo los mensajes de una cierta persona.
 \item Buscar una cadena de texto sólo en los mensajes de un determinado hilo.
 \item Descargar el hilo como un fichero, o cualquier otra forma de exportar la
 información para poder acceder a ella desde un cliente de correo electrónico
 o fuera de línea.
 \item Responder a un mensaje usando un cliente de correo (o un webmail) y 
 citando el mensaje original.
\end{itemize}

Al indexar los archivos de las listas de correo, los buscadores se encuentran
en ocasiones que los mensajes están replicados en varios servidores (mirrors).
Al no tener forma de identificar los mensajes, la desgraciada consecuencia
es que los mensajes aparecen varias veces en los resultados de las búsquedas,
y sólo el usuario puede darse cuenta de que se trata de una repetición.
Naturalmente, el comportamiento ideal sería que los mensajes aparecieran
sólo una vez en los resultados del buscador.

\section*{Origen de los problemas: pérdida de información}

En el origen de estos problemas se encuentra una pérdida de información
que se produce al convertir los mensajes archivados a HTML para su publicación
en la web.

Los gestores más habituales de listas de correo (mailman, majordomo, sympa,
etc.) generan un fichero en formato Mailbox (mbox) con los mensajes que
han sido enviados a la lista.

Otros programas independientes, como hypermail, monharc, pipermail..., se
especializan en convertir el fichero Mailbox en un conjunto de páginas
web estáticas.

Los programas más sofisticados son capaces de generar índices complejos
de los archivos (por fecha, por autor, por hilo...), con múltiples referencias
cruzadas entre los mensajes en forma de hipervínculos (mensaje anterior,
mensaje siguiente, etc.).

Pero incluso en el mejor de los casos, esta información sólo es comprensible
para el usuario, nunca para la máquina.

En consecuencia, es imposible explotarla más allá de las formas previstas
por el programa que ha generado los archivos.

Entre la información que se pierde en la publicación, se encuentra:

\begin{itemize}
 \item El asunto del mensaje.
 \item El autor del mensaje.
 \item La fecha del mensaje.
 \item La referencia a la lista de correo en la que se publicó el mensaje.
 \item La referencia al mensaje anterior, si existe.
 \item Las referencias (enlaces) a las posibles respuestas al mensaje.
\end{itemize}

\section*{Propuesta para conservar la información}

Las tecnologías de la web semántica (y concretamente, RDF) son perfectamente
capaces de publicar en la web toda la información señalada en la sección
anterior. Dado que la información ya existe en el origen, no es necesario 
ningún procedimiento manual para enriquecerla. Tan sólo debe considerarse un 
proceso de conversión que no desprecie la información, sino que la publique 
junto con los archivos en HTML. De esta forma, las listas de correo se 
introducirían en la web semántica.

\section*{Aplicaciones}

Enriquecer semánticamente la publicación web de los archivos de las listas
de correo abriría la puerta a nuevas aplicaciones:

\begin{itemize}
  \item Eliminar la aparición repetida de los mismos mensajes en los resultados
 	de los buscadores. Para lograrlo, los buscadores deberían procesar la 
	información semántica para reconocer las copias (mirrors) de los archivos.
  \item Implementar en los navegadores nuevas funcionalidades para resolver alguno
	de los problemas antes señalados. Estas capacidades, que mejorarían 
	sensiblemente la comodidad en la consulta de los archivos, podrían añadirse 
	como extensiones o plug-ins de los navegadores actuales.
  \item Obtener información sobre los suscriptores de una lista de correo. Por 
	ejemplo, conocer en qué otras listas de correo participa una persona. 
	Esta aplicación es especialmente interesante en conexión con 
	FOAF\footnote{\url{http://www.foaf-project.org/}.} De este modo, se 
	podría sacar una \emph{orla} con las fotos de los participantes en una 
	lista de correo \footnote{Como hace GNOME, véase 
	\url{http://planet.gnome.org/heads/}}, o situarlos geográficamente en un 
	mapa \footnote{Como hace Debian, 
	véase \url{http://www.debian.org/devel/developers.loc}}
  \item Facilitar la internacionalización. Al hacer comprensibles las relaciones entre 
	los mensajes por el software, el navegador proporcionaría las opciones de 
	exploración (mensaje siguiente, mensaje anterior, etc.) en el idioma del 
	usuario, independientemente del idioma en el que se encontrasen las páginas 
	HTML.
  \item Mejorar la accesibilidad de la información. Las tecnologías de accesibilidad 
	podrían informar sobre quién es el autor del mensaje o cuántas respuestas hay, 
	usando la voz u otros medios.
\end{itemize}

\section*{Estado del arte}

Existen algunos trabajos similares a esta propuesta:

\begin{itemize}
  \item El proyecto DOAML\footnote{\url{http://www.doaml.net/}} consiste en un 
	vocabulario RDF para describir listas de correo. Como ejemplo, en 
	la web del proyecto se encuentran las descripciones de las listas 
	de correo del W3C. La información de este vocabulario limita sus 
	referencias a los mensajes archivados a un enlace a la versión HTML 
	de éstos.
  \item Por otro lado, EMiR\footnote{\url{http://xmlns.filsa.org/emir/}} es un 
	esquema RDF para describir mensajes de correo electrónico. 
  \item En la misma línea se encuentra XMTP\footnote{\url{http://www.openhealth.org/xmtp/}}.
\end{itemize}

\section*{Conclusiones}

Introducir los archivos de las listas de correo en la web semántica sólo
requiere disponer de una aplicación de publicación que utilice la tecnología
apropiada (RDF) como complemento al HTML.

Con un mínimo esfuerzo, los administradores de todas las listas de correo
podrían emplear la aplicación en sus listas, por lo que la implantación
sería rápida\footnote{En realidad, cualquier suscriptor (no necesariamente 
el administrador) de una lista de correo podría publicar los archivos enriquecidos.
Tan sólo debería disponer de todos los mensajes antiguos almacenados en su 
cliente de correo electrónico, y exportarlos al formato Mbox.}. Además, al 
no requerirse la participación de un experto para el enriquecimiento de la 
información, resultaría posible enriquecer inmediatamente grandes volúmenes 
de información, incluso listas de correo que lleven muchos años en funcionamiento.

El desarrollo de una aplicación de estas características requeriría, en
primer lugar, la creación de un esquema de información, que muy bien podría
ser una combinación de los ya existentes; y en segundo lugar, el procesamiento 
de un fichero Mbox para extraer la información que contiene.


\section{Conclusiones personales}

Este proyecto me ha supuesto muchas cosas a nivel personal y profesional. He
aprendido mucho sobre esta incipiente área que es la Web Semántica; tengo la
suerte de contar con dos directores de proyecto (José Emilio Labra y Diego
Berrueta) que son grandes expertos en la materia y me han ayudado en todos 
estos largos meses.

Además me ha brindado la oportunidad de aprender un nuevo lenguaje de programación 
por el que hacia tiempo tenia curiosidad: Python. Las conclusiones sobre este 
maravilloso lenguaje no pueden ser mejores.

Desde el punto de vista ético para mi era muy importante saber que las horas
invertidas en este proyecto no se \emph{morirían} en las polvorientas estanterías
de la biblioteca. Por eso desde eso desde el principio y en todo momento del
proceso de desarrollo todos los componentes de este proyecto (código y 
documentación) han estado disponibles en el subversion de 
Berlios\footnote{\url{http://swaml.berlios.de/wsvn}} de manera totalmente
libre.

En estos meses he liberado media docena de versiones de SWAML. Quizás el recorrido
termine aquí, o quizás no. No sé si yo continuaré desarrollando SWAML o si a alguien
le parecerá interesante para continuar el trabajo que yo he comenzado; pero el caso
es que el software liberado ahí seguirá para cualquiera que sea el uso que alguien
le quiera dar.

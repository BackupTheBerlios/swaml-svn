\documentclass[spanish]{beamer}

\usetheme{Warsaw}
\usepackage{beamerthemesplit}

\usepackage[spanish]{babel}
\usepackage[utf8]{inputenc}
\usepackage{listings}
\usepackage{graphicx}
\usepackage{colortbl}

\title{SWAML}
\subtitle{publicaci\'on de listas de correo en Web Sem\'antica}
\author{Sergio Fern\'andez L\'opez}
\institute{%
Proyecto Fin de Carrera\\
E.U. de Ingenier\'ia T\'ecnica en Inform\'atica de Oviedo
}
\date{20 de Diciembre de 2006}

\begin{document}

\frame{\titlepage}

%\frame{\tableofcontents}

\section{Introducción}

\subsection{Situación actual}
\frame
{
  \frametitle{Publicación actual de los archivos de listas de correo}

  \begin{itemize}
  \item<1-> Miles de listas de correo de las más variopinta temática
  \item<2-> Publicación en HTML de los archivos antiguos
  \item<3-> Pérdida de toda posibilidad de recuperar esa información
  \end{itemize}
}

\subsection{Objetivos}
\frame
{
  \frametitle{Objetivos}

  \begin{enumerate}
  \item Publicación de los archivos antiguos de listas de correo en un formato rico semánticamente
  \item FIXME
  \end{enumerate}
}

\subsection{La Web Semántica}
\frame
{
  \frametitle{Introducción a la Web Semántica}

  FIXME
}

\section{Meollo}

\subsection{Tecnologías implicadas}
\frame
{
  \frametitle{RDF}

  FIXME
}
\frame
{
  \frametitle{SIOC}

  FIXME
}
\frame
{
  \frametitle{Python}

  FIXME
}

\subsection{Meollo}
\frame
{
  \frametitle{Meollo}

  FIXME
}

\section{Demostración}
\subsection{Demostración}
\frame
{
  \begin{center}
    \LARGE{\textbf{demostración práctica}}
  \end{center}
}
\subsection{Preguntas}
\frame
{
  \begin{center}
    \LARGE{\textbf{¿preguntas?}}
  \end{center}
}

\end{document}


En la elaboraci�n del presente proyecto se ha seguido, se forma totalmente
natural, \emph{eXtreme Programming}\footnote{\url{http://www.extremeprogramming.org/}} 
(Programaci�n Extrema) como metodolog�a de trabajo.

\section{Justificaci�n}

Las circunstacias porfesionales del alumno en el periodo que se realiz� el
proyecto, obligaban a usar alguna metodolog�a �gil que permitiera un �ptimo y
flexible reparto de tiempo entre todas las tareas a desarrollar.

As� se termin� utilizando de manera natural muchos de las normas de este m�todo:

\begin{itemize}
  \item Simplicidad: desde el primer momentos se busc� hacer el proyecto de la 
	manera m�s sencilla posible.
  \item Desarrollo iterativo e incremental: una vez se fue teniendo las 
	funcionalidades principales, se fueron construyendo alrededor peque�as 
	utilidades para enriquecer la aplicaci�n.
  \item Propiedad del c�digo compartida: desde las primeras lineas, todo el 
	c�digo ha estado publicado en el 
	Subversion\footnote{\url{http://subversion.tigris.org/}}  del proyecto, 
	al alcance en cualquier momento de las tres personas involucradas en el 
	proyecto.
  \item Refactorizaci�n del c�digo: de manera frecuente partes del c�digo del 
	proyecto han sido sometidas a t�cnicas de refactorizaci�n para mejorar 
	partes que no hab�a sido codificadas de la mejor de las maneras.
  \item Pruebas unitarias (FIXME: mentira, pero es una tarea que tengo que abordar 
	con PyUnit)
  \item Frecuente interacci�n del equipo de programaci�n con el cliente o usuario:
	en este caso la figura del cliente ha tenido la forma del co-director del
	proyecto (Diego Berrueta), con el que se han mantenido frecuentes encuentros
	para tratar aspectos concretos del proyecto.
\end{itemize}

Aunque tambi�n ha habido una t�cnica que no se ha podido poner en practica: la
programaci�n por parejas o pair programming. Si bien es verdad que ha habido muchas
tareas que han desarrollado dos personas, no ha sido una programaci�n por parejas 
como tal.


\section{Introducci�n a la Programaci�n Extrema}

FIXME(�diego?)

\chapter{An�lisis}

\section{An�lisis tecnol�gico}

\subsection{Formato de salida}

RDF, FIXME

\subsection{Lenguaje de programaci�n}

El problema planteado requiere de una lenguaje de programaci�n que disponga
de determinadas caracter�sticas:

\begin{itemize}
  \item F�cil despliegue: hay que procurar que SWAML se pueda desplegar en
	todo tipo de m�quinas, sin excesivos requisitos ni hardware ni software.
  \item API para RDF: que disponga de una bliblioteca, a poder ser nativa, para
	manejar RDF.
  \item Biblioteca para ficheros mbox: ser�a interesante disponer de una biblioteca 
	que abstraiga lo mayor posible al proyecto del manejo de ficheros
	mbox\footnote{\url{http://rfc.net/rfc4155.html}} y mensajes de correo 
	electr�nico\footnote{\url{http://rfc.net/rfc2822.html}}.
\end{itemize}

Y todas esas razones encuentran en 
Python\footnote{\url{http://www.python.org/}} una justificaci�n:

\begin{itemize}
  \item Python es un lenguaje de script extremadamente eficiente. Su uso est� 
	muy extendido en todos los sistemas Unix actuales (GNU/Linux, familia 
	BSD, Solaris, etc), aunque tambi�n est� 
	disponible\footnote{\url{http://www.python.org/download/}}
	para la mayoria de sistemas operativos restantes 
	(Windows\footnote{url{http://www.python.org/download/windows/}}, 
	MacOs\footnote{\url{http://www.python.org/download/mac/}} y
	otros\footnote{\url{http://www.python.org/download/other/}}.
  \item Existen varias posibilidades para manejar RDF desde Python. Algunas son
	bibliotecas nativas desarrolladas tambi�n en Python, y otras est�n
	disponibles en forma de bindings a bibliotecas desarrolladas en otro 
	lenguaje.
	De todas las posibilidades, quiz�s RDFLib\footnote{\url{http://rdflib.net/}}
	sea la que se encuentra en un estado de desarrollo m�s avanzado y maduro.
	Adem�s ofrece la posibilidad de \emph{colocar encima} otras bibliotecas,
	como por ejemplo Sparta\footnote{\url{http://www.mnot.net/sw/sparta/}},
	para utilizar determinados conceptos que no contempla RDFLib.
  \item Con modulos como mailbox\footnote{\url{http://docs.python.org/lib/module-mailbox.html}}
	e email\footnote{\url{http://docs.python.org/lib/module-email.html}} esta
	necesidad se ve claramente satisfecha.
\end{itemize}

Por tanto se concluy� que Python era una de las mejores opciones para resolver
el problema planteado.

\section{Casos de uso}

FIXME

\section{Descripci�n de actores}

FIXME

\section{Escenarios}

FIXMR
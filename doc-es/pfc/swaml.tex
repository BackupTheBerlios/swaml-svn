\documentclass[spanish,a4paper,10pt]{report}
\usepackage[latin1]{inputenc}
\usepackage[T1]{fontenc}
\usepackage[margin=1in]{geometry}
\usepackage{paralist}
\usepackage{graphicx}
\usepackage{url}
\usepackage{times}
\usepackage{babel}
\usepackage{listings}
\usepackage{verbatim}


\pagestyle{headings}

\bibliographystyle{plain}

\title{%
	SWAML, Publicaci�n de listas de correo en Web Sem�ntica}
\author{Sergio Fern\'andez L\'opez,
	Diego Berrueta Mu\~noz,
	Jos� Emilio Labra Gayo}
\date{Julio de 2006}

\begin{document}

\maketitle

\chapter*{Resumen}

Este documento contiene la documentaci�n del proyecto titulado
\emph{SWAML, publicaci�n de listas de correo en Web Sem�ntica}, 
presentado por el autor para la obtenci�nn del t�tulo de Ingeniero 
T�cnico en Inform�tica por la Escuela Universitaria de Ingenieria 
T�cnica en Inform�tica de Oviedo (Universidad de Oviedo).

El objetivo del proyecto es publicar listas de correo en formatos
con sem�ntica (principalmente RDF), para investigar y completar
determinada informaci�n que no es posible conseguir con los formatos
de publicaci�n actuales.

La p�gina web del proyecto es \url{http://swaml.berlios.de/}.

\section*{Palabras clave}

Web Sem�ntica, RDF, OWL, Python, mbox.

\chapter*{Licencia}

El contenido de este documento se encuentra licenciado bajo la licencia 
Creative Commons Reconocimiento-CompartirIgual 2.5 (anexo \ref{sec:license.cc}), 
a excepci�n del c�digo fuente (anexo \ref{sec:source}) que se encuentra licenciado 
bajo la licencia GNU General Public License (GPL) (anexo \ref{sec:license.gpl}), 
versi�n 2 o superior.

\chapter*{Historial de este documento}

\begin{tabular}{|l|l|l|}
\hline
Fecha & Versi�n & Comentarios \\\hline
Jul/2006 & 0.0.1 & Primer borrador \\\hline
\end{tabular}

\tableofcontents

\newpage

\listoffigures

\newpage

\chapter{Introducci�n}

\section{Introducci�n}

Los archivos de las listas de correo (es decir, los mensajes antiguos) son
frecuentemente publicados en la web e indexados por los buscadores convencional
es. La base de conocimientos que introducen en la web es enorme.

Sin embargo, una gran cantidad de informaci�n se pierde durante la publicaci�n,
con el resultado de que los archivos publicados son inc�modos de consultar
y poco funcionales.

Este art�culo describe la aplicaci�n de la web sem�ntica para evitar la
p�rdida de informaci�n y habilitar la construcci�n de nuevas aplicaciones
para explotar m�s convenientemente la informaci�n.

\section{Propuesta}

Las listas de correo son parte fundamental de la comunicaci�n en Internet.
Existen listas de correo dedicadas a cualquier tema de inter\'es imaginable.
Hoy en d�a, es com�n que las listas de correo publiquen sus archivos (los
mensajes antiguos) en forma de p�ginas web, lo que dispara su utilidad,
especialmente en combinaci�n con los buscadores actuales. Gracias a esta 
publicaci�n, es posible consultar los mensajes desde el navegador, sin 
necesidad de estar suscrito a las listas de correo, y tambi�n se puede 
localizar un mensaje usando Google u otro buscador.

Estos archivos contienen una formidable base de conocimiento, especialmente
en temas t�cnicos. Un uso muy com�n consiste en introducir un mensaje de error 
(de una aplicaci�n) en Google\footnote{http://www.google.es/} y obtener como 
resultado un mensaje archivado que aborda el problema, probablemente porque 
alguien se ha encontrado previamente con el mismo error y ha efectuado la 
consulta en una lista de correo p�blica. Con suerte, alguna de las respuestas 
al mensaje localizado contendr� la soluci�n al problema, aportada por un experto 
suscrito a la lista de correo.

\subsection{Problemas}

Por desgracia, consultar los archivos de una lista de correo en la web es
m�s inc�modo que hacerlo mediante un cliente de correo electr�nico.
Por poner s�lo algunos ejemplos, el navegador no permite ejecutar ninguna
de estas acciones:

\begin{itemize}
 \item Mostrar el hilo de la conversaci�n en forma de �rbol.
 \item Imprimir el hilo completo.
 \item Mostrar una lista de los mensajes entre dos fechas arbitrarias.
 \item Ocultar los mensajes que no tienen respuestas.
 \item Mostrar s�lo los mensajes de una cierta persona.
 \item Buscar una cadena de texto s�lo en los mensajes de un determinado hilo.
 \item Descargar el hilo como un fichero, o cualquier otra forma de exportar la
 informaci�n para poder acceder a ella desde un cliente de correo electr�nico
 o fuera de l�nea.
 \item Responder a un mensaje usando un cliente de correo (o un webmail) y 
 citando el mensaje original.
\end{itemize}

Al indexar los archivos de las listas de correo, los buscadores se encuentran
en ocasiones que los mensajes est�n replicados en varios servidores (mirrors).
Al no tener forma de identificar los mensajes, la desgraciada consecuencia
es que los mensajes aparecen varias veces en los resultados de las búsquedas,
y s�lo el usuario puede darse cuenta de que se trata de una repetici�n.
Naturalmente, el comportamiento ideal ser�a que los mensajes aparecieran
s�lo una vez en los resultados del buscador.

\subsection{Origen de los problemas: p�rdida de informaci�n}

En el origen de estos problemas se encuentra una p�rdida de informaci�n
que se produce al convertir los mensajes archivados a HTML para su publicaci�n
en la web.

Los gestores m�s habituales de listas de correo (mailman, majordomo, sympa,
etc.) generan un fichero en formato Mailbox (mbox) con los mensajes que
han sido enviados a la lista.

Otros programas independientes, como hypermail, monharc, pipermail..., se
especializan en convertir el fichero Mailbox en un conjunto de p�ginas
web est�ticas.

Los programas m�s sofisticados son capaces de generar �ndices complejos
de los archivos (por fecha, por autor, por hilo...), con múltiples referencias
cruzadas entre los mensajes en forma de hiperv�nculos (mensaje anterior,
mensaje siguiente, etc.).

Pero incluso en el mejor de los casos, esta informaci�n s�lo es comprensible
para el usuario, nunca para la m�quina.

En consecuencia, es imposible explotarla m�s all� de las formas previstas
por el programa que ha generado los archivos.

Entre la informaci�n que se pierde en la publicaci�n, se encuentra:

\begin{itemize}
 \item El asunto del mensaje.
 \item El autor del mensaje.
 \item La fecha del mensaje.
 \item La referencia a la lista de correo en la que se public� el mensaje.
 \item La referencia al mensaje anterior, si existe.
 \item Las referencias (enlaces) a las posibles respuestas al mensaje.
\end{itemize}

\subsection{Propuesta para conservar la informaci�n}

Las tecnolog�as de la web sem�ntica (y concretamente, RDF) son perfectamente
capaces de publicar en la web toda la informaci�n se�alada en la secci�n
anterior. Dado que la informaci�n ya existe en el origen, no es necesario 
ningún procedimiento manual para enriquecerla. Tan s�lo debe considerarse un 
proceso de conversi�n que no desprecie la informaci�n, sino que la publique 
junto con los archivos en HTML. De esta forma, las listas de correo se 
introducir�an en la web sem�ntica.

\section{Introducci�n a la Web Sem�ntica}

En 1989 Tim Berners-Lee realiz� para el CERN un modelo de gesti�n de la informaci�n
basado en un sistema distribuido de hipertexto. Fue el origen de lenguaje de marcado
HTML y la semila de la Web actual.

No tard� mucho en darse cuenta que ese modelo no era suficiente para manejar grandes
vol�menes de informaci�n, pues resultar�a muy dificil encontrarla y usarla eficientemente.

\subsection{Estructura de la Web Sem�ntica}

En la figura \ref{fig:swStack} encontramos el dise�o preliminar de la web sem�ntica:

\begin{figure}[h]
	\centering
	\includegraphics[width=12cm]{images/semantic-web-stack.png}
	\caption{Pila de la web sem�ntica}
	\label{fig:swStack}
\end{figure}

\subsection{RDF}

Acr�nimo del ingl�s \emph{Resource Description Framework} (marco de descripci�n de recursos). 

\begin{figure} [h]
\begin{verbatim}
<rdf:RDF xmlns:rdf="http://www.w3.org/1999/02/22-rdf-syntax-ns#"
	 xmlns:rdfs="http://www.w3.org/2000/01/rdf-schema#"
	 xmlns:dc="http://purl.org/dc/elements/1.1/"
>
  <rdf:Description rdf:about="http://www.wikier.org/">
    <dc:creator>Sergio Fdez</dc:creator>
    <rdfs:seeAlso rdf:resource="http://www.wikier.org/foaf.rdf"/>	
  </rdf:Description>
</rdf:RDF>
\end{verbatim}
	\caption{Ejemplo de grafo RDF serializado en XML.}
	\label{fig:ejemplo.rdfxml}
\end{figure}

\subsection{Publicaciones}

Aunque no existen buenas publicaciones en castellano, hay dos libros en ingl�s que
es imprescindible ojear: \emph{A Semantic Web primer}\cite{SemanticWebPrimer} y
\emph{Practical RDF}\cite{OreillyPracticalRDF}.



\chapter{Introducci�n a la Web Sem�ntica}
\input{intro-web-semantica.tex}

\chapter{Memoria}
\input{memoria.tex}

\chapter{Planificaci�n}
\input{planificacion.tex}

\chapter{Metodolog�a}

En la elaboraci�n del presente proyecto se ha seguido, se forma totalmente
natural, \emph{eXtreme Programming}\footnote{\url{http://www.extremeprogramming.org/}} 
(Programaci�n Extrema) como metodolog�a de trabajo.

\section{Justificaci�n}

Las circunstacias porfesionales del alumno en el periodo que se realiz� el
proyecto, obligaban a usar alguna metodolog�a �gil que permitiera un �ptimo y
flexible reparto de tiempo entre todas las tareas a desarrollar.

As� se termin� utilizando de manera natural muchos de las normas de este m�todo:

\begin{itemize}
  \item Simplicidad: desde el primer momentos se busc� hacer el proyecto de la 
	manera m�s sencilla posible.
  \item Desarrollo iterativo e incremental: una vez se fue teniendo las 
	funcionalidades principales, se fueron construyendo alrededor peque�as 
	utilidades para enriquecer la aplicaci�n.
  \item Propiedad del c�digo compartida: desde las primeras lineas, todo el 
	c�digo ha estado publicado en el 
	Subversion\footnote{\url{http://subversion.tigris.org/}}  del proyecto, 
	al alcance en cualquier momento de las tres personas involucradas en el 
	proyecto.
  \item Refactorizaci�n del c�digo: de manera frecuente partes del c�digo del 
	proyecto han sido sometidas a t�cnicas de refactorizaci�n para mejorar 
	partes que no hab�a sido codificadas de la mejor de las maneras.
  \item Pruebas unitarias (FIXME: mentira, pero es una tarea que tengo que abordar 
	con PyUnit)
  \item Frecuente interacci�n del equipo de programaci�n con el cliente o usuario:
	en este caso la figura del cliente ha tenido la forma del co-director del
	proyecto (Diego Berrueta), con el que se han mantenido frecuentes encuentros
	para tratar aspectos concretos del proyecto.
\end{itemize}

Aunque tambi�n ha habido una t�cnica que no se ha podido poner en practica: la
programaci�n por parejas o pair programming. Si bien es verdad que ha habido muchas
tareas que han desarrollado dos personas, no ha sido una programaci�n por parejas 
como tal.


\section{Introducci�n a la Programaci�n Extrema}

FIXME(�diego?)

\chapter{Manuales}

\chapter{Manuales}

En esta documentaci�n se incluyen dos manuales:

\begin{itemize}
 \item Manual t�cnico (incluye el c�digo fuente en un anexo).
 \item Manual de usuario.
\end{itemize}

\section{Manual t�cnico}

\section{Manual de usuario}


\chapter{Anexos}

\chapter{Anexos}

\section{Ontolog�a}

FIXME(incluir swaml.owl)

\section{C�digo fuente\label{sec:source}} 

FIXME(automatizar)

\newpage

\section{Licencias} 

FIXME(ojear \url{http://sciencecommons.org/literature/scholars_copyright})

\subsection{Creative Commons Reconocimiento-CompartirIgual 2.5\label{sec:license.cc}}

\input{licenses/cc-2.5.tex}

\newpage

\subsection{GNU General Public License (GPL)\label{sec:license.gpl}}

\input{licenses/gpl-2.0.tex}

\newpage

\section{Acerca}

Este documento ha sido escrito usando \LaTeX.


\bibliography{bibliografia}

\end{document}

